\chapter{Introduction}
\label{cha:intro}
The first contains a general introduction to the work. The goals are defined and the modus operandi is explained.

\section{Related work}
\subsection{Paper 1}
\textbf{Quantifying flexibility in EV charging as DR potential: Analysis of two real-world data sets}

(I have marked some sentences in this paper which could be useful for my abstract or introduction. Also, some of the plots are very descriptive and could be useful as well for the presentation slides.)

In this paper, their goals are to (1) characterize the EV charging behaviour and collect those behaviours into three clusters (park to charge, charging near home and charging near work), (2) to fit statistical models for the characteristics of each cluster (soujourn times and flexibility), and (3) quantify the maximal aggregate load that could be attained by coordinating connected EV charging at a given time \emph{t}, until time $t+\Delta$.

Our problem is encompassed into a local charge near work environment, so I have extracted the results for this cluster in detail. The charge near work cluster has the following characteristics:

\begin{itemize}
  \item Arrivals: Early morning.
  \item Departures: Late afternoon.
  \item Soujourns: Average around 9h.
  \item Resulting flexibility: Mostly on weekdays and during daytime.
\end{itemize}

In order to compare our density distributions from EnergyVille data to this paper's distributions, keep in mind the following values. Sojourn times [min, max] values are [5.00, 18.52] and idle times values are [0, 15.54]. Fitted distributions are logistic so better to check the rest of parameters directly if simulation is needed.


\section{Lorem Ipsum 6--7}
\lipsum[6-7]

%%% Local Variables: 
%%% mode: latex
%%% TeX-master: "thesis"
%%% End: 
